%%
%% Author: jsalazer
%% 8/21/2019
%%

%% TODO:

% Preamble
\documentclass[12pt]{article}
% Use latex macro
\input latexmacro
\standard

\oddsidemargin=0in%
\evensidemargin=0in %
\topmargin =-.5in %
\textheight=8.5in %
\textwidth=6.4in %

% Packages
\usepackage{amsmath}
\usepackage{natbib}
\usepackage{graphicx}
\usepackage{tikz-network}

\usetikzlibrary{shapes}

\pagenumbering{arabic}

\graphicspath{{./img/}}


% Document
\begin{document}
\double

{\Large \bf Sets and Parameters}

Let $G=(V, A)$ be a directed graph where $|V|=n+2$,
and the depot is represented by the two vertices $0$ and $n+1$.
Feasible vehicle routes then correspond to paths starting at vertex $0$ and ending at vertex $n+1$.

Let $N$ be set of non-depot locations where $N=V\{0, n+1\}$.

Let $A$ be the set of arcs where $A=\{(i, j)\st i, j \in V, i \neq j\}$.
Let $\delta^{+}(i)=\{j\st (i, j) \in A\}$
and $\delta^{-}(j)=\{i\st (i, j) \in A\}$. Note that arc $(i,j) \in A$ denotes
location $j$ is visited after location $i$.

Let $K$ be the set of vehicles where $|K|=m$. Let $\bar{s}_i$ and $s_i$ be the service time
at node $i\in V$ without and with helper, respectively.
Note that $\bar{s}_{0}=\bar{s}_{n+1}=0$ and $s_0=s_{n+1}=0$.

Let $t_{ij}$ be the travel minutes from location $i\in V$ to location $j\in V\setminus\{i\}$.

Let $[a_i, b_i]$ be time window for non-depot location $i\in V\setminus\{0, n+1\}$ where
$a_i$ and $b_i$ are starting and ending times. Let $[a_0, b_0]$ and $[a_{n+1}, b_{n+1}]$ be
time windows for the depot. We can define $[a_0, b_0]$ and $[a_{n+1}, b_{n+1}]$ as

\begin{equationarray}{lll}
  a_0&=&\min_{i\in N}\{a_i-t_{0i}\}, \nonumber\\[10pt]
  b_0&=&\max_{i\in N}\{b_i-t_{0i}\}, \nonumber\\[10pt]
  a_{n+1}&=&\min_{i\in N}\{a_i + s_{i} + t_{in+1}\}, \nonumber\\[10pt]
  b_{n+1}&=&\max_{i\in N}\{b_i+s_i+t_{in+1}\}. \nonumber
\end{equationarray}

Let $q_i$ denote the demand of location $i$ and let $Q_k$ be the vehicle $k\in K$ capacity.

Let $c_k$, $h_k$, and $p_k$ be unit transportation, helper, and team truck costs for vehicle $k \in K$.

\newpage
{\Large \bf Variables}

Let $x_{ijk}$ be the binary variable for sequencing where

\begin{equationarray}{rlllll}
    &
    x_{ijk} &
    = &
    \left\{
    \begin{array}{ll}
        1, &
        \mbox{if and only if arc } (i, j)\in A \mbox{ is used by vehicle } k \in K, \nonumber \\[5pt]
        0, &
        \mbox{otherwise.}
    \end{array}
    \right. &
\end{equationarray}

Let $w_{ik}$ be continuous variable indicating the time at which vehicle $k \in K$ starts servicing location $i \in V$.

Let $y_{k}$ be the binary variable for helper assignment where

\begin{equationarray}{rlllll}
    &
    y_{k} &
    = &
    \left\{
    \begin{array}{ll}
        1, &
        \mbox{if vehicle } k \in K \mbox{ has a helper}, \nonumber \\[5pt]
        0, &
        \mbox{otherwise.}
    \end{array}
    \right. &
\end{equationarray}

Let $z_{k}$ be the binary variable for team driver assignment where

\begin{equationarray}{rlllll}
    &
    z_{k} &
    = &
    \left\{
    \begin{array}{ll}
        1, &
        \mbox{if vehicle } k \in K \mbox{ has a team driver}, \nonumber \\[5pt]
        0, &
        \mbox{otherwise.}
    \end{array}
    \right. &
\end{equationarray}


\newpage
{\Large \bf Mathematical Formulation}

The model is formulated as,

\begin{equationarray}{rrrlll}
    \mbox{minimize} \sum_{k\in K}\Big[\sum_{(i,j)\in A}c_kt_{ij}x_{ijk} + h_ky_k + q_kz_k\Big]
    \label{model: objective}
\end{equationarray}

\vspace{-6pt}

\begin{equationarray}{rrrlll}

    \mbox{subject to}
    &
    \sum_{k \in K} \sum_{j \in \delta^{+}(i)}x_{ijk} &
    = & 1, &
    \forall i\in N, \label{model: each customer is visited by once}\\[18pt]

    &
    \sum_{j \in \delta^{+}(0)}x_{0jk} &
    = & 1, &
    \forall k\in K, \label{model: each vehicle should leave the depot}\\[18pt]

    &
    \sum_{j \in \delta^{-}(i)}x_{jik} - \sum_{j \in \delta^{+}(i)}x_{ijk} &
    = & 0, &
    \forall k\in K, \forall i \in N, \label{model: flow conservation}\\[18pt]

    &
    \sum_{i \in \delta^{-}(n+1)}x_{in+1k} &
    = & 1, &
    \forall k\in K, \label{model: each vehicle should arrive the depot}\\[18pt]


    &
    y_{ij} &
    \in\ &
    \{0,1\} &
    \forall i\in N_3, j \in N_4,
    \label{model:rdc-1: cdc allocation binary} \\[18pt]

    &
    z_i &
    \in\ &
    \{0,1\} &
    \forall i\in N_3.
    \label{model:rdc-1: rdc site selection binary}

\end{equationarray}


\newpage

We now provide the mathematical formulation for the RDC Network Design problem.

Let $x_{ijpbt}$ be the variable for the flow of product in pounds from node $i \in N$ to node $j \in N\setminus\{i\}$ for product $p\in P$ in BU $b \in B$ at period
$t \in T$.

Let $z_i$ be the binary variable for the selection of an RDC node $i \in N_3$ where

\begin{equationarray}{rlllll}
    &
    z_{i} &
    = &
    \left\{
    \begin{array}{ll}
        1, &
        \mbox{if RDC node } i \in N_3 \mbox{ is selected}, \nonumber \\[5pt]
        0, &
        \mbox{otherwise.}
    \end{array}
    \right. &
\end{equationarray}


The model \textbf{RDC-1} is formulated as,

\begin{equationarray}{rrrlll}
    \mbox{minimize} \sum_{i \in \{N_2 \cup N_3\}} \sum_{j \in \{N_3 \setminus\{i\} \cup N_4\}} \sum_{p \in P} \sum_{b \in B} \sum_{t \in T} \bar{c}_{ijpb}x_{ijpbt}
    \label{model:rdc-1: objective}
\end{equationarray}

\vspace{-6pt}

\begin{equationarray}{rrrlll}

    \mbox{subject to} &
    \sum_{i \in N_3} x_{ijpbt} &
    = &
    d_{jptbt}, &
    \forall j\in N_4, p\in P, b \in B, t\in T,
    \label{model:rdc-1: demand}\\[18pt]

    &
    \sum_{i \in \{N_1 \cup N_2\}} x_{ijpbt} &
    = &
    \sum_{i \in N_4} x_{jipbt}, &
    \forall j \in N_3, p \in P, b \in B, t \in T,
    \label{model:rdc-1: flow in flow out}\\[18pt]

    &
    \sum_{i \in \{N_1 \cup N_2\}} \sum_{p \in P} \sum_{b \in B} \ell_p * \frac{x_{ijpbt}}{u_{ipt}} &
    \le &
    v_{jt}, &
    \forall j \in N_3, t \in T,
    \label{model:rdc-1: rdc capacity}\\[18pt]

    &
    \sum_{j \in N_3} \sum_{b \in B} x_{ijpbt} &
    \le &
    w_{ipt}, &
    \forall i \in N_2, p \in P, t \in T,
    \label{model:rdc-1: rp capacity}\\[18pt]

    &
    \sum_{j \in N_4} y_{ij} &
    = &
    1, &
    \forall i \in N_3,
    \label{model:rdc-1: cdc allocation}\\[18pt]

    &
    \sum_{p \in P} \sum_{t \in T} \sum_{b \in B} x_{ijpbt} &
    \le &
    M * y_{ij}, &
    \forall i \in N_3, j \in N_4,
    \label{model:rdc-1: cdc allocation big M}\\[18pt]

    &
    \sum_{j \in N_4} y_{ij} &
    \le &
    M * z_i, &
    \forall i \in N_3,
    \label{model:rdc-1: rdc selection big M}\\[18pt]

    &
    x_{ijpbt} &
    \ge &
    0, &
    \forall i \in N, j\in \{N \setminus \{i\}\}, \nonumber \\&
    &
    &
    &
    p \in P, t \in T, b \in B,
    \label{model:rdc-1: variable positivity real} \\[18pt]

    &
    y_{ij} &
    \in\ &
    \{0,1\} &
    \forall i\in N_3, j \in N_4,
    \label{model:rdc-1: cdc allocation binary} \\[18pt]

    &
    z_i &
    \in\ &
    \{0,1\} &
    \forall i\in N_3.
    \label{model:rdc-1: rdc site selection binary}

\end{equationarray}

The objective of the model, (\ref{model:rdc-1: objective}), minimizes the total transportation costs from RP to RDC to customer.
In (\ref{model:rdc-1: demand}), total product demand for each customer should be satisfied for each period. Flow balances at the
RDC are described in (\ref{model:rdc-1: flow in flow out}). The RDC capacity is measured in total pallets, and thus
(\ref{model:rdc-1: rdc capacity}) describes the conversion of the flow in pounds into pallets. In (\ref{model:rdc-1: rp capacity}), the
flow out of the RP is constrained by its capacity for the horizon. RDC assignments to CDC are described
in (\ref{model:rdc-1: cdc allocation}) and (\ref{model:rdc-1: cdc allocation big M}), which ensures one and only CDC is assigned
to a RDC for the model horizon. The model is able to decide whether an RDC is open or closed for the model horizon
which is described in (\ref{model:rdc-1: rdc selection big M}). Lastly, nonegativity and binary constraints are described (\ref{model:rdc-1: variable positivity real}),
(\ref{model:rdc-1: cdc allocation binary}), and (\ref{model:rdc-1: rdc site selection binary}).

\end{document}